\input{preamble}
\begin{document}

\title{Интерференция лазерного излучения}
\thanks{4.5.2}

\input{name}

\begin{abstract}
Цель работы: исследовать зависимость видности интерференционной картины от угла $\beta$ между плоскостями поляризации интерферирующих волн при нулевой разности хода, зависимость видности интерференционной картины от разности хода интерферирующих пучков для угла $\beta = 0$. По результатам измерений следует оценить спектральные характеристики лазерного излучения: ширину спектра генерации и число генерируемых мод.

\end{abstract}

\pacs{Valid PACS appear here}% PACS, the Physics and Astronomy
                             % Classification Scheme.
%\keywords{Suggested keywords}%Use showkeys class option if keyword
                              %display desired
\maketitle

%\tableofcontent

\begin{enumerate}

    \item 
    Исследуем зависимость видности интерференционной картины от угла $\beta$ поворота первого поляроида при нулевой разности хода $(\nu_2 = 1)$: включим блок питания фотодиода и измерим величины $h_1, h_2, h_3, h_4$ на экране осциллографа. 
    
    \item
    Исследуем зависимость видности от разности хода между пучками. Для этого установим первый поляроид в положение, в котором интерференционная картина видна наиболее чётко $(\beta~=~0$\textdegree$, \nu_3 = 1)$. Снимем зависимость величин $h_1, h_2, h_3, h_4$ от координаты $x$ второго блока, начиная с минимального расстояния $(x = 12~cm)$.
    
    \item
    Рассчитаем коэффициент $\nu_3$. Построим графики $\nu_3(cos(\beta))$ и $\nu_3(cos(\beta)^2)$. 
    
    \begin{figure}[h]
    \center{\includegraphics[scale=0.17]{my_plot1.png}}
    \end{figure}
    
    Легко видеть, что с точностью определяемой систематической ошибкой проведённых измерений теоретическая гипотеза о линейной зависимости показателя видности от $|cos(\beta)|$ выполняется.
    
    \begin{figure}[h]
    \center{\includegraphics[scale=0.17]{my_plot2.png}}
    \end{figure}
    
    \item
    Рассчитаем коэффициент $\nu_2$. Построим график зависимости видности $\nu_2(x)$ от координаты второго блока. Определим по графику расстояния между максимумами, оценим расстояние $L$ между зеркалами оптического резонатора лазера и межмодовое расстояние $\Delta \nu_m$. 
    
    \begin{figure}[h]
    \center{\includegraphics[scale=0.17]{my_plot3.png}}
    \end{figure}
    
    \begin{gather*}
        L = (30 \pm 5)~\text{cm}\\
        \Delta \nu_m = (5 \pm 1)~10^{8}~\text{s}^{-1}\\
    \end{gather*}

    \item
    Определим задержку $l_{1/2}$ (полуширину) на половине высоты главного максимума и рассчитаем диапазон частот $\Delta \nu_{\text{полн}}$, в котором происходит генерация продольных мод. Оценим число генерируемых лазером продольных мод.
    
    \begin{gather*}
        l_{1/2} = (10 \pm 2)~\text{cm}\\
        \Delta \nu_{\text{полн}} \approx \frac{0.6 c}{l_1
        {1/2}} = (1.8 \pm 0.4)~\text{s}^{-1}\\
        n \approx (5 \pm 1)\\
    \end{gather*}

    Стоит заметить, что все расчёты несут исключительно оценочный характер и верны только в границах обозначенных погрешностей. 
    
    
\end{enumerate}

\end{document}

