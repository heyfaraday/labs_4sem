\input{preamble}
\begin{document}

\title{Дифракция света на ультразвуковой волне в жидкости}
\thanks{4.3.2}

\input{name}

\begin{abstract}
Цель работы: изучение дифракции света на синусоидальной акустической решётке и наблюдение фазовой решетки методом тёмного поля.
В работе используются: оптическая скамья, осветитель, два длиннофокусных объектива, кювета с жидкостью, кварцевый излучатель с микрометрическим винтом, генератор ультразвуковой частоты, линза, вертикальная нить на рейтере, микроскоп.

\end{abstract}

\pacs{Valid PACS appear here}

\maketitle

\begin{enumerate}

    \item 
    
    \textbf{Определение скорости ультразвука по дифракционной картине.}\\
    
    а) Соберём схему. Включим осветитель. Максимально откроем входную щель.

    б) Поместив лист бумаги между коллиматором и кюветой, убедимся, что световое пятно равномерно освещено; затем проверим освещённость пятна на выходе из прибора.
    
    в) Настроим микроскоп и отсчётное устройство. Установим рабочую ширину щели 20–30 мкм.
    
    Получим дифракционную картину. Картина видна наиболее чётко, когда в кювете образуется стоячая УЗ-волна. 
    
    Определим положения дифракционных полос. Повторим измерения для трёх–четырёх фиксированных частот, лежащих в одном из диапазонов. 
    
    Построим на одном листе графики $Y = Y(m)$ (от $−m$ до $+m$). 
    \begin{figure}[h]
    \center{\includegraphics[scale=0.17]{my_plot1.png}}
    \end{figure}
    
    Для каждой частоты определим по наклону прямой расстояние между соседними полосами (цена деления микрометрического винта - 4 мкм). 
    \begin{gather*}
        b_{\nu_1} = (118 \pm 4)~10^{-6}~m,\\
        b_{\nu_2} = (107 \pm 4)~10^{-6}~m,\\
        b_{\nu_3} = (122 \pm 4)~10^{-6}~m.
    \end{gather*}
    
    Зная фокусное расстояние объектива ($f$ = 28 см) и полосу пропускания красного фильтра ($\lambda = 6400 \pm 200$ \AA), рассчитаем длину УЗ-волны $\Lambda$.
    \begin{gather*}
        \Lambda_{\nu_1} = (14 \pm 2)~10^{-4}~m,\\
        \Lambda_{\nu_2} = (17 \pm 3)~10^{-4}~m,\\
        \Lambda_{\nu_3} = (15 \pm 2)~10^{-4}~m.
    \end{gather*}
    
    Рассчитаем скорость звука для каждой частоты и среднюю скорость.
    \begin{gather*}
        v_{\nu_1} = (1600 \pm 100)~m / s,\\
        v_{\nu_2} = (1900 \pm 200)~m / s,\\
        v_{\nu_3} = (1800 \pm 200)~m / s.
    \end{gather*}
    
    \begin{gather*}
        v_{mean} = (1800 \pm 200)~m / s,\\
    \end{gather*}
    
    
    \item 

    \textbf{Определение скорости ультразвука методом тёмного поля.}\\
    
    Настроим установку на метод тёмного поля.
    
    Закроем нулевой дифракционный максимум проволочкой. Меняя частоту, пронаблюдаем акустическую решётку. Убедимся, что при удалении проволочки с главного максимума решётка не видна.
    
    Зафиксируем с помощью окулярной шкалы микроскопа координаты первой и последней из хорошо видимых в поле зрения тёмных полос и количество светлых промежутков между ними.
    
    Повторим измерения для нескольких частот внутри одной из рабочих полос УЗ-излучателя.
    
    \begin{figure}[h]
    \center{\includegraphics[scale=0.17]{my_plot2.png}}
    \end{figure}
    
    Для каждой частоты рассчитаем длину УЗ-волны $\Lambda$ с учётом удвоения числа наблюдаемых полос.
    
    Построим график $\Lambda = F(1/\nu)$ и определим по наклону прямой скорость ультразвука в воде.
    \begin{gather*}
        b = v = (2600 \pm 200)~m / s.
    \end{gather*}
    
    Табличное значение скорости звука в воде при температуре $20^{\circ}С$:
    \begin{gather*}
        v_{\text{табл}} = 1481~m / s.
    \end{gather*}

\end{enumerate}

\end{document}
